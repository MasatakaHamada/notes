\documentclass{jsarticle}
\usepackage{amsmath, amssymb} % 数学記号のパッケージ

% 定理などの番号付けに関するパッケージとその設定
\usepackage{amsthm}
\theoremstyle{definition}
\newtheorem{qst}{問題}
\newtheorem{rem}[qst]{注意}
\renewcommand\proofname{\bf 解答} % proof環境の設定変更

\begin{document}

\begin{qst}
実数列$\{a_n\}_{n=1}^\infty$に対して
\[ \lim_{n\to\infty}\frac{a_n}{n}=0\Rightarrow\lim_{n\to\infty}a_n=0 \]
は成り立つか.成り立つ場合は証明し,成り立たない場合は反例を挙げよ.
\end{qst}
\begin{proof}
成り立たない.
実際,数列$\{1\}_{n=1}^\infty$が反例である.
\end{proof}

\begin{qst}
$f\colon\mathbb{R}\to\mathbb{R}$に対して
\[ f は有界かつ連続 \Rightarrow f は一様連続 \]
は成り立つか.成り立つ場合は証明し,成り立たない場合は反例を挙げよ.
\end{qst}
\begin{proof}
成り立たない.
実際,$f(x)=\sin{x^2}$とすると,$f$は有界かつ連続であるが,
任意の$\delta>0$に対して$n>\delta^{-2}$を満たす$n\geq1$をとれば,
$x=\sqrt{n\pi}, y=\sqrt{n\pi+\pi/2}$のとき
\[ |x-y|=\sqrt{n\pi+\frac{\pi}{2}}-\sqrt{n\pi}
=\frac{\pi/2}{\sqrt{n\pi+\pi/2}+\sqrt{n\pi}}
<\frac{\pi/2}{2\sqrt{n\pi}}<\frac{2}{2\sqrt{n\pi}}
=\frac{1}{\sqrt{n\pi}}<\frac{1}{\sqrt{n}}
<\delta \]
かつ
\[ |\sin{x^2}-\sin{y^2}|=\left|\sin{n\pi}-\sin\left(n\pi+\frac{\pi}{2}\right)\right|=1 \]
となるから,$f$は一様連続でない.
\end{proof}

\begin{qst}
実数列$\{a_n\}_{n=1}^\infty$および$a\in\mathbb{R}$に対して
\[ \lim_{n\to\infty}a_n=a\Leftrightarrow\{a_n\}_{n=1}^\infty の任意の部分列が a に収束する部分列を持つ \]
が成り立つことを証明せよ.
\end{qst}
\begin{proof}
($\Rightarrow$)
$\{a_n\}_{n=1}^\infty$の任意の部分列が$a$に収束することから従う.

($\Leftarrow$)
対偶を証明する.
すなわち,ある$\varepsilon>0$が存在して,任意の$N\geq1$に対して
$n\geq N$かつ$|a_n-a|\geq\varepsilon$を満たす$n\geq1$が存在すると仮定する.
各$N\geq1$に対して,$n\geq N$かつ$|a_n-a|\geq\varepsilon$を満たす$n\geq1$を
ひとつ選び$n(N)$とおく.さらに,$A:=\{n(N):N\geq1\}$とし,
数列$\{k(m)\}_{m=1}^\infty$を次のように定める:
$k(1):=\min A$,$m\geq1$に対し$k(m+1):=\min(A\setminus\{k(1),\cdots,k(m)\})$.
このとき,部分列$\{a_{k(m)}\}_{m=1}^\infty$はすべての$m\geq1$に対して
$|a_{k(m)}-a|\geq\varepsilon$を満たすから,$a$に収束する部分列を持たない.
\end{proof}

\begin{qst}
$\mathbb{N}$の空でない任意の部分集合$A$は最小値を持つことを証明せよ.
\end{qst}
\begin{proof}
(1)
$A$が有限集合であるとき,$n:=\sharp{A}$に関する帰納法で証明する.
$n=1$のときは$a\in A$が最小値となり,
$n=2$のときは,相異なる$a,b\in A$に対して$\min\{a,b\}$が最小値となる.
$n=k\geq2$のとき,$A$の最小値が存在すると仮定する.
$\sharp{A}=k+1$とすると,$a\in A$をひとつ選んで
$A=\{a\}\cup B$,$\sharp{B}=k$と表せる.
このとき$a\leq\min B$ならば$\min A=a$,そうでなければ$\min A=\min B$となる.
よって,$n=k+1$のときも$A$の最小値が存在する.

(2)
$A$が無限集合であるとき,
$n\in A$をひとつ取り$A_n:=\{a\in A:a<n\}$とおく.
$A_n=\varnothing$ならば$n=\min A$である.
$A_n\neq\varnothing$のとき,$A_n$は有限集合であるから
(1)より$\min A_n$が存在し,$\min A=\min A_n$である.
\end{proof}

\begin{qst}
$x\in\mathbb{R}$に対して
\[ \delta_x(B):=1_B(x)=\begin{cases}1 & (x\in B) \\ 0 & (x\notin B)\end{cases},B\in\mathcal{B}(\mathbb{R}) \]
と定める.このとき,$\delta_x$は$(\mathbb{R},\mathcal{B}(\mathbb{R}))$上の確率測度であることを示せ.
なお,この$\delta_x$は$x$を台とするディラック測度と呼ばれる.
\end{qst}
\begin{proof}
定義より$\delta_x(\mathbb{R})=1$かつ$B\in\mathcal{B}(\mathbb{R})$に対して$\delta_x(B)\geq0$である.
$B_1,B_2,\cdots\in\mathcal{B}(\mathbb{R})$を$i\neq j$に対して$B_i\cap B_j=\varnothing$を満たすものとする.
すべての$i$に対して$x\notin B_i$であるとき,
\[ \delta_x\left(\bigcup_{i=1}^\infty B_i\right)=0=\sum_{i=1}^\infty\delta_x(B_i) \]
である.また,ある$N$が存在して$x\in B_N$であるとき,$B_1,B_2,\cdots$の定義よりそのような$N$はただひとつであるから
\[ \delta_x\left(\bigcup_{i=1}^\infty B_i\right)=1=\delta_x(B_N)=\sum_{i=1}^\infty\delta_x(B_i) \]
である.
\end{proof}

\begin{qst}
$f\colon[0,\infty)\to\mathbb{R}$を$(0,\infty)$上連続な関数とする.このとき
\[ \lim_{r\to+0, r\in\mathbb{Q}}f(r)=f(0) \Rightarrow \lim_{x\to+0}f(x)=f(0) \]
が成り立つことを証明せよ(従って,$f$は$[0,\infty)$上連続となる).
\end{qst}
\begin{proof}
$\varepsilon>0$とする.
仮定より,ある$\delta>0$が存在して,任意の$r\in(0,\delta)\cap\mathbb{Q}$に対して
$|f(r)-f(0)|<\varepsilon/2$が成り立つ.
また,任意の$x\in(0,\delta)$に対して,$f$の連続性より,
ある$\delta'>0$が存在して,$|y-x|<\delta'$を満たす$y\in(0,\delta)$に対し
$|f(y)-f(x)|<\varepsilon/2$が成り立つ.
一方,$|r-x|<\delta'$を満たす$r\in(0,\delta)\cap\mathbb{Q}$が存在する.
よって
$|f(x)-f(0)|\leq|f(x)-f(r)|+|f(r)-f(0)|<\varepsilon/2+\varepsilon/2=\varepsilon$
となるから,結論が従う.
\end{proof}
\begin{proof}[\textbf{誤答例}]
0に収束する任意の非負実数列$\{x_n\}_{n=1}^\infty$に対して,
$\{f(x_n)\}_{n=1}^\infty$が$f(0)$に収束することを示す.
そのためには,$\{f(x_n)\}_{n=1}^\infty$の任意の部分列が
$f(0)$に収束する部分列を有することを示せばよい.
そこで,$\{f(x_k)\}_{k=1}^\infty$を$\{f(x_n)\}_{n=1}^\infty$の任意の部分列とする.
このとき,
\textbf{$\{x_k\}_{k=1}^\infty$から有理数である項を取り出して得られる部分列を$\{x_{k_l}\}_{l=1}^\infty$とする}と,
仮定より$\{f(x_{k_l})\}_{l=1}^\infty$は$f(0)$に収束する.
よって,$\{f(x_k)\}_{k=1}^\infty$に対して$f(0)$に収束する部分列が得られたので,結論が従う.
\end{proof}
\begin{proof}[\textbf{誤答例}]
$\varepsilon>0$とする.
仮定より,ある$\delta_1>0$が存在して,$0<r<\delta_1$を満たす任意の$r\in\mathbb{Q}$に対して
$|f(r)-f(0)|<\varepsilon/2$を満たす.
以下,このような$\delta_1$を固定し,さらに$0<r<\delta_1$を満たす$r\in\mathbb{Q}$を固定する.
$f$は$(0,\infty)$上連続であるから,ある$\delta_2>0$が存在して,
$|x-r|<\delta_2$を満たす任意の$x\in(0,\infty)$に対して$|f(x)-f(r)|<\varepsilon/2$を満たす.
このとき,\textbf{$\delta:=\min\{\delta_2-\delta_1,0\}$とおく}と,
$0<x<\delta$を満たす任意の$x\in(0,\infty)$に対して
$|x-r|\leq x+r<\delta+\delta_1\leq\delta_2$となる.よって
$|f(x)-f(0)|\leq|f(x)-f(r)|+|f(r)-f(0)|<\varepsilon/2+\varepsilon/2=\varepsilon$である.
従って,結論が成り立つ.
\end{proof}

\begin{qst}
区間$I$上の連続関数列$\{f_n\}_{n=1}^\infty$と
$I$上の連続関数$f$に対して,次の2つの条件は同値であることを示せ.
\begin{enumerate}
\item$\{f_n\}_{n=1}^\infty$は$n\to\infty$のとき$I$上$f$に一様収束する.
\item$\displaystyle\sup_{x\in I}|f_n(x)-f(x)|$は$n\to\infty$のとき0に収束する.
\end{enumerate}
\end{qst}
\begin{proof}
(1$\Rightarrow$2)
仮定より,任意の$\varepsilon>0$に対してある$N$が存在して,
$n\geq N$ならば任意の$x\in I$に対して$|f_n(x)-f(x)|<\varepsilon$が成り立つ.
よって,$n\geq N$ならば$\displaystyle\sup_{x\in I}|f_n(x)-f(x)|<\varepsilon$となる.
従って,$\displaystyle\sup_{x\in I}|f_n(x)-f(x)|$は$n\to\infty$のとき0に収束する.

(2$\Rightarrow$1)
仮定より,任意の$\varepsilon>0$に対してある$N$が存在して,
$n\geq N$ならば$\displaystyle\sup_{x\in I}|f_n(x)-f(x)|<\varepsilon$が成り立つ.
よって,$n\geq N$ならば任意の$x\in I$に対して
$|f_n(x)-f(x)|\leq\displaystyle\sup_{x\in I}|f_n(x)-f(x)|<\varepsilon$が成り立つ.
従って,$\{f_n\}_{n=1}^\infty$は$n\to\infty$のとき$I$上$f$に一様収束する.
\end{proof}

\begin{qst}
$\|\cdot\|$をノルムに持つバナッハ空間$V$上のコーシー列$\{v_n\}$と
任意の$\varepsilon>0$に対して,
\[ m,n\geq N\Rightarrow\|v_m-v_n\|<\varepsilon \]
を満たす自然数$N$を選ぶ.$\{v_n\}$の極限を$v$とするとき,
\[ n\geq N\Rightarrow\|v-v_n\|\leq\varepsilon \]
が成り立つことを証明せよ.
\end{qst}
\begin{proof}
任意の$\varepsilon'>0$に対して,
$m\geq M$ならば$\|v_m-v\|<\varepsilon'$となるような自然数$M$が存在する.
よって,$m,n\geq\max\{M,N\}$ならば
$\|v-v_n\|=\|(v_m-v_n)+(v-v_m)\|\leq\|v_m-v_n\|+\|v_m-v\|<\varepsilon+\varepsilon'$
が成り立つ.
ゆえに,$n\geq N$ならば$\|v-v_n\|<\varepsilon+\varepsilon'$が
任意の$\varepsilon'>0$に対して成り立つ.
このとき,もし$\|v-v_n\|>\varepsilon$であるとすると,
$\varepsilon<\delta<\|v-v_n\|$なる実数$\delta$が存在する.
この$\delta$に対して$\|v-v_n\|>\delta=\varepsilon+(\delta-\varepsilon)$となり,
$\|v-v_n\|<\varepsilon+\varepsilon'$が
任意の$\varepsilon'>0$に対して成り立つことに矛盾する.
従って,$n\geq N\Rightarrow\|v-v_n\|\leq\varepsilon$が成り立つ.
\end{proof}

\begin{qst}
実数列$\{a_n\}$と$a\in\mathbb{R}$に対して,次の2つの条件は同値であることを示せ.
\begin{enumerate}
\item$\displaystyle\lim_{n\to\infty}a_n=a$である.
すなわち,任意の$\varepsilon>0$に対してある$N\in\mathbb{N}$が存在して,
$n\geq N$ならば$|a_n-a|<\varepsilon$が成り立つ.
\item 任意の$\varepsilon>0$に対してある$N\in\mathbb{N}$が存在して,
$n\geq N$ならば$|a_n-a|\leq\varepsilon$が成り立つ.
\end{enumerate}
\end{qst}
\begin{proof}
(1$\Rightarrow$2)
$|a_n-a|<\varepsilon$ならば$|a_n-a|\leq\varepsilon$であることから従う.

(2$\Rightarrow$1)
$\varepsilon>0$とする.
2より,ある$N\in\mathbb{N}$が存在して,
$n\geq N$ならば$|a_n-a|\leq\varepsilon/2$が成り立つ.
このとき,$\varepsilon/2<\varepsilon$より$|a_n-a|<\varepsilon$である.
\end{proof}

\begin{qst}
集合$X$および$X$の部分集合列$A_1,A_2,\cdots$に対して次が成り立つことを示せ.
\[ 1_{\limsup_{n\to\infty}A_n}(x)=\limsup_{n\to\infty}1_{A_n}(x),
1_{\liminf_{n\to\infty}A_n}(x)=\liminf_{n\to\infty}1_{A_n}(x) \]
\end{qst}
\begin{proof}
以下
$F=\displaystyle\limsup_{n\to\infty}A_n$,
$G=\displaystyle\liminf_{n\to\infty}A_n$
とおく.
$x\in F$のとき,任意の$m\geq1$に対してある$n\geq m$が存在して$x\in A_n$を満たすから,
\[ \limsup_{n\to\infty}1_{A_n}(x)=\inf_{m\geq1}\sup_{n\geq m}1_{A_n}(x)=1 \]
である.
$x\notin F$のとき,ある$m\geq1$が存在して$n\geq m$ならば$x\notin A_n$となるから,
\[ \limsup_{n\to\infty}1_{A_n}(x)=\inf_{m\geq1}\sup_{n\geq m}1_{A_n}(x)=0 \]
である.
$x\in G$のとき,ある$m\geq1$が存在して$n\geq m$ならば$x\in A_n$となるから,
\[ \liminf_{n\to\infty}1_{A_n}(x)=\sup_{m\geq1}\inf_{n\geq m}1_{A_n}(x)=1 \]
である.
$x\notin G$のとき,任意の$m\geq1$に対してある$n\geq m$が存在して$x\notin A_n$を満たすから,
\[ \liminf_{n\to\infty}1_{A_n}(x)=\sup_{m\geq1}\inf_{n\geq m}1_{A_n}(x)=0 \]
である.以上より
\[ 1_{\limsup_{n\to\infty}A_n}(x)=\limsup_{n\to\infty}1_{A_n}(x),
1_{\liminf_{n\to\infty}A_n}(x)=\liminf_{n\to\infty}1_{A_n}(x) \]
が成り立つ.
\end{proof}

\begin{qst}
$X,X_1,X_2,\cdots$を2乗可積分な確率変数とする.
$X_n$が$X$に2次平均収束するとき,$X_n^2$の期待値は$X^2$の期待値に収束することを示せ.
すなわち
\[ \lim_{n\to\infty}E[(X_n-X)^2]=0\Rightarrow\lim_{n\to\infty}E[X_n^2]=E[X^2] \]
であることを示せ.
\end{qst}
\begin{proof}
$E[(X_n-X)^2]\to0$より$\|X_n-X\|_2\to0$であるから
$|\|X_n\|_2-\|X\|_2|\leq\|X_n-X\|_2\to0$
ゆえ$\|X_n\|_2\to\|X\|_2$すなわち$E[X_n^2]\to E[X^2]$が従う.
\end{proof}

\begin{qst}
$(\Omega,\mathcal{F},P)$を確率空間とする.
$A,B\in\mathcal{F}$に対して,$P(A)=1$ならば$P(A\cap B)=P(B)$であることを示せ.
\end{qst}
\begin{proof}
$P(A)=1$より$P(A^c)=0$であるから,
特に$P(A^c\cap B)\leq P(A^c)=0$より$P(A^c\cap B)=0$である.
よって$P(B)=P(A\cap B)+P(A^c\cap B)=P(A\cap B)$となる.
\end{proof}

\begin{qst}
$(\Omega,\mathcal{F},P)$を確率空間,$X$を可積分確率変数とし,
$A\in\mathcal{F}$は$P(A)=1$を満たすとする.
このとき$E(X)=E(X;A)$が成り立つことを示せ.
\end{qst}
\begin{proof}
$P(A)=1$より$P(A^c)=0$であり,零集合上の積分は0であるから$E(X;A^c)=0$である.
よって$E(X)=E(X;A)+E(X;A^c)=E(X;A)$となる.
\end{proof}

\begin{qst}
$(X,\mathcal{M})$を可測空間とする.
$X$の空でない部分集合$A$に対して
\[ \mathcal{M}_A=\{A\cap M:M\in\mathcal{M}\} \]
と定めるとき,$(A,\mathcal{M}_A)$は可測空間であることを示せ.
\end{qst}
\begin{proof}
$A\cap\varnothing=\varnothing$,
$A\cap X=A$ゆえ$\varnothing,A\in\mathcal{M}_A$である.
また,$F=A\cap M\in\mathcal{M}_A$のとき,
$A=(A\cap M)\cup(A\cap M^c)$であることに注意すると
$A\setminus F=A\setminus(A\cap M)=A\cap M^c$ゆえ
$A\setminus F\in\mathcal{M}_A$である.
さらに,$n=1,2,\cdots$に対して$F_n=A\cap M_n\in\mathcal{M}_A$とするとき,
$\bigcup_nF_n=\bigcup_n(A\cap M_n)=A\cap\bigcup_nM_n$ゆえ
$\bigcup_nF_n\in\mathcal{M}_A$である.
以上より,$(A,\mathcal{M}_A)$は可測空間である.
\end{proof}

\begin{qst}
$(\Omega,\mathcal{F},P)$を確率空間とする.
$p\in(0,1)$とし,$\{\xi_n\}_{n=1,2,\cdots}$を独立同分布な確率変数列で
$P(\xi_1=1)=p=1-P(\xi_1=-1)$を満たすものとする.
また,$\mathcal{F}_0=\{\varnothing,\Omega\}$,
$\mathcal{F}_n=\sigma(\xi_1,\cdots,\xi_n)$とおく.
さらに$k\in\mathbb{Z}$とし,$\{S_n\}_{n=0,1,\cdots}$を
$S_0=k$,$S_n=k+\xi_1+\cdots+\xi_n$により定める.
このとき,$\{S_n\}_{n=0,1,\cdots}$が
$\{\mathcal{F}_n\}_{n=0,1,\cdots}$に関して
マルチンゲールであるための$p$に関する必要十分条件を求めよ.
\end{qst}
\begin{proof}
$\{S_n\}_{n=0,1,\cdots}$および$\{\mathcal{F}_n\}_{n=0,1,\cdots}$の定義より,
各$n$に対して$S_n$は$\mathcal{F}_n$-可測である.また
\[ E(|S_n|)\leq k+\sum_{i=1}^nE(|\xi_i|)
=k+nE(|\xi_1|)=k+n<\infty \]
より,各$n$に対して$S_n$は可積分である.
さらに,$S_{n+1}-S_n=\xi_{n+1}$であり,
$\xi_{n+1}$は$\mathcal{F}_n$と独立であるから
\[ E(S_{n+1}-S_n\mid\mathcal{F}_n)=
E(\xi_{n+1}\mid\mathcal{F}_n)=E(\xi_{n+1})
=1\cdot p+(-1)(1-p)=2p-1 \]
となる.
$\{S_n\}_{n=0,1,\cdots}$が
$\{\mathcal{F}_n\}_{n=0,1,\cdots}$に関してマルチンゲールならば
$E(S_{n+1}-S_n\mid\mathcal{F}_n)=0$
であるから,上式より$p=1/2$である.
逆に$p=1/2$ならば上式より$E(S_{n+1}-S_n\mid\mathcal{F}_n)=0$であるから,
$\{S_n\}_{n=0,1,\cdots}$は
$\{\mathcal{F}_n\}_{n=0,1,\cdots}$に関してマルチンゲールとなる.
以上より,$\{S_n\}_{n=0,1,\cdots}$が
$\{\mathcal{F}_n\}_{n=0,1,\cdots}$に関して
マルチンゲールであるための$p$に関する必要十分条件は$p=1/2$である.
\end{proof}

\begin{qst}
$(\Omega,\mathcal{F},P)$を確率空間とする.
$\{\xi_n\}_{n=1,2,\cdots}$を独立同分布な確率変数列で,
$k=1,2,\cdots$に対して
\[ P(\xi_1=k)=P(\xi_1=-k)=\frac{1}{2^{k+1}} \]
を満たすものとする.
また,$\mathcal{F}_0=\{\varnothing,\Omega\}$,
$\mathcal{F}_n=\sigma(\xi_1,\cdots,\xi_n)$とおく.
さらに,$\{S_n\}_{n=0,1,\cdots}$を$S_0=0$,$S_n=\xi_1+\cdots+\xi_n$により定める.
このとき,$\{S_n\}_{n=0,1,\cdots}$は
$\{\mathcal{F}_n\}_{n=0,1,\cdots}$に関してマルチンゲールであることを示せ.
\end{qst}
\begin{proof}
$\{S_n\}_{n=0,1,\cdots}$および$\{\mathcal{F}_n\}_{n=0,1,\cdots}$の定義より,
各$n$に対して$S_n$は$\mathcal{F}_n$-可測である.また
\[ E(|S_n|)\leq\sum_{i=1}^nE(|\xi_i|)
=nE(|\xi_1|)=n\sum_{k=1}^\infty\frac{k}{2^k}<\infty
(\because\text{ratio test}) \]
より,各$n$に対して$S_n$は可積分である.
さらに,$S_{n+1}-S_n=\xi_{n+1}$であり,
$\xi_{n+1}$は$\mathcal{F}_n$と独立であるから
\[ E(S_{n+1}-S_n\mid\mathcal{F}_n)=
E(\xi_{n+1}\mid\mathcal{F}_n)=E(\xi_{n+1})
=\sum_{k=1}^\infty\frac{k}{2^{k+1}}+\sum_{k=1}^\infty\frac{-k}{2^{k+1}}
=0 \]
となる.
従って,$\{S_n\}_{n=0,1,\cdots}$は
$\{\mathcal{F}_n\}_{n=0,1,\cdots}$に関してマルチンゲールである.
\end{proof}

\begin{qst}
$(\Omega,\mathcal{F},P)$を確率空間とする.
確率変数$X$が可測となるような最小の$\sigma$-加法族を$\sigma(X)$とするとき,
$\sigma(X)=\{X^{-1}(B):B\in\mathcal{B}\}$であることを示せ.
\end{qst}
\begin{proof}
$\mathcal{G}=\{X^{-1}(B):B\in\mathcal{B}\}$は$\sigma$-加法族である.実際
\begin{itemize}
\item$\varnothing=X^{-1}(\varnothing)\in\mathcal{G}$,$\Omega=X^{-1}(\mathbb{R})\in\mathcal{G}$,
\item$G=X^{-1}(B)\in\mathcal{G}$のとき$G^c=X^{-1}(B^c)\in\mathcal{G}$,
\item$n\in\mathbb{N}$に対して$G_n=X^{-1}(B_n)\in\mathcal{G}$のとき
$\bigcup_nG_n=X^{-1}(\bigcup_nB_n)\in\mathcal{G}$
\end{itemize}
である.$\mathcal{G}$の定義より,任意の$B\in\mathcal{B}$に対して
$X^{-1}(B)\in\mathcal{G}$,すなわち$X$は$\mathcal{G}$-可測であるから
$\sigma(X)\subset\mathcal{G}$が成り立つ.
また,$X$は$\sigma(X)$-可測であるから,任意の$B\in\mathcal{B}$に対して
$X^{-1}(B)\in\sigma(X)$であり,ゆえに$\mathcal{G}\subset\sigma(X)$である.
従って$\sigma(X)=\mathcal{G}$である.
\end{proof}

\begin{qst}
$(\Omega,\mathcal{F},P)$を確率空間とする.
確率変数$X_1,\cdots,X_n$が可測となるような
最小の$\sigma$-加法族を$\sigma(X_1,\cdots,X_n)$とするとき,
次が成り立つことを示せ.
\[ \sigma(X_1,\cdots,X_n)=\sigma\left(\bigcup_{k=1}^n\sigma(X_k)\right) \]
\end{qst}
\begin{proof}
定義より,任意の$k=1,\cdots,n$に対して
$\sigma(X_k)\subset\sigma(X_1,\cdots,X_n)$が成り立つから
\[ \bigcup_{k=1}^n\sigma(X_k)\subset\sigma(X_1,\cdots,X_n) \]
である.これより
\[ \sigma\left(\bigcup_{k=1}^n\sigma(X_k)\right)\subset\sigma(X_1,\cdots,X_n) \]
を得る.また,任意の$B_k\in\mathcal{B}$($k=1,\cdots,n$)に対して
\[ X_k^{-1}(B_k)\in\sigma(X_k)\subset\bigcup_{k=1}^n\sigma(X_k)
\subset\sigma\left(\bigcup_{k=1}^n\sigma(X_k)\right) \]
が成り立つから,$X_1,\cdots,X_n$は
$\displaystyle\sigma\left(\bigcup_{k=1}^n\sigma(X_k)\right)$-可測である.
これより
\[ \sigma(X_1,\cdots,X_n)\subset\sigma\left(\bigcup_{k=1}^n\sigma(X_k)\right) \]
を得る.以上より
\[ \sigma(X_1,\cdots,X_n)=\sigma\left(\bigcup_{k=1}^n\sigma(X_k)\right) \]
が成り立つ.
\end{proof}

\begin{qst}
$(\Omega,\mathcal{F},P)$を確率空間とする.
独立な確率変数列$\{X_n\}_{n=1}^\infty$に対して
$\mathcal{F}_n=\sigma(X_1,\cdots,X_n)$($n=1,2,\cdots$)とおくとき,
$X_{n+1}$は$\mathcal{F}_n$と独立であることを示せ.
\end{qst}
\begin{proof}
$\mathcal{F}_n$は$\pi$-系$\bigcup_{k=1}^n\sigma(X_k)$が生成する$\sigma$-加法族である.
また,$\sigma(X_{n+1})$は$\sigma$-加法族であるから特に$\pi$-系である.
よって,$\pi$-$\lambda$定理より,
$F\in\sigma(X_{n+1})$および$G\in\bigcup_{k=1}^n\sigma(X_k)$に対して
$P(F\cap G)=P(F)P(G)$であることを示せばよい.
ところが,$\{X_n\}_{n=1}^\infty$は独立であるから,$P(F\cap G)=P(F)P(G)$は直ちに従う.
ゆえに,$X_{n+1}$は$\mathcal{F}_n$と独立である.
\end{proof}

\begin{qst}
確率変数列$\{X_n\}_{n=1}^\infty$に対して次が成り立つことを示せ.
\[ \sup_n\|X_n\|_2<\infty\Leftrightarrow\sup_nE(X_n^2)<\infty \]
\end{qst}
\begin{proof}
($\Rightarrow$)
$x^2$は$x\geq0$において単調非減少であるから
$E(X_n^2)=\|X_n\|_2^2\leq(\sup\|X_n\|_2)^2$が成り立つ.
従って$\sup E(X_n^2)\leq(\sup\|X_n\|_2)^2<\infty$である.

($\Leftarrow$)
$\sqrt{x}$は$x\geq0$において単調非減少であるから
$\|X_n\|_2=\sqrt{E(X_n^2)}\leq\sqrt{\sup E(X_n)^2}$が成り立つ.
従って$\sup\|X_n\|_2\leq\sqrt{\sup E(X_n)^2}<\infty$である.
\end{proof}

\begin{qst}
$0\leq x\leq y$に対して$\sqrt{x}\leq\sqrt{y}$であることを示せ.
\end{qst}
\begin{proof}
$x=y=0$のときは明らかに成り立つ.
$x,y$の少なくとも一方が0でないとき,
$0\leq y-x=(\sqrt{y}+\sqrt{x})(\sqrt{y}-\sqrt{x})$の両辺を
$\sqrt{y}+\sqrt{x}>0$で割れば$0\leq\sqrt{y}-\sqrt{x}$となり,結論を得る.
\end{proof}

\begin{qst}
実数列$\{a_n\}_{n=1}^\infty,\{b_n\}_{n=1}^\infty$がそれぞれ
実数$a,b$に収束するとき,以下の問いに答えよ.
\begin{itemize}
\item[(1)]任意の$n=1,2,\cdots$に対し$a_n<b_n$が成り立つならば$a\leq b$であることを示せ.
\item[(2)](1)において$a<b$は成り立つか.成り立つ場合は証明し,成り立たない場合は反例を挙げよ.
\end{itemize}
\end{qst}
\begin{proof}
\begin{itemize}
\item[(1)]
実数列$\{a_n\}_{n=1}^\infty,\{b_n\}_{n=1}^\infty$がそれぞれ
実数$a,b$に収束するから,
任意の$\varepsilon>0$について適当な$N\in\mathbb{N}$をとれば,
$n\geq N$に対して
$|a_n-a|<\frac{\varepsilon}{2}$,$|b_n-b|<\frac{\varepsilon}{2}$
が成り立つ.よって
\[
a-b=a-a_N-b+a_N<a-a_N-b+b_N\leq|a_N-a|+|b_N-b|<
\frac{\varepsilon}{2}+\frac{\varepsilon}{2}=\varepsilon
\]
となり,任意の$\varepsilon>0$に対し$a-b<\varepsilon$である.
従って$a\leq b$が成り立つ.
\item[(2)]
成り立たない.実際,
$a_n=-\frac{1}{n}$,$b_n=\frac{1}{n}$
とすると,任意の$n=1,2,\cdots$に対し$a_n<b_n$が成り立つが,
$\{a_n\}_{n=1}^\infty,\{b_n\}_{n=1}^\infty$はともに0に収束する.
\end{itemize}
\end{proof}

\begin{qst}
$\mathcal{F},\mathcal{G}$を集合$X$上の$\sigma$-加法族とするとき,次の問いに答えよ.
\begin{itemize}
\item[(1)]$\mathcal{F}\cap\mathcal{G}$は$X$上の$\sigma$-加法族であることを示せ.
\item[(2)]$\mathcal{F}\cup\mathcal{G}$は$X$上の$\sigma$-加法族か.
そうである場合は証明し,そうでない場合は反例を挙げよ.
\end{itemize}
\end{qst}
\begin{proof}
\begin{itemize}
\item[(1)]
$\mathcal{F},\mathcal{G}$はともに$X$上の$\sigma$-加法族であるから,
\begin{itemize}
\item$X\in\mathcal{F}$かつ$X\in\mathcal{G}$より$X\in\mathcal{F}\cap\mathcal{G}$,
\item$A\in\mathcal{F}\cap\mathcal{G}$ならば
$A\in\mathcal{F}$かつ$A\in\mathcal{G}$より
$X\setminus A\in\mathcal{F}$かつ$X\setminus A\in\mathcal{G}$
すなわち$X\setminus A\in\mathcal{F}\cap\mathcal{G}$,
\item$A_1,A_2,\cdots\in\mathcal{F}\cap\mathcal{G}$ならば
$A_1,A_2,\cdots\in\mathcal{F}$かつ$A_1,A_2,\cdots\in\mathcal{G}$より
$\bigcup_nA_n\in\mathcal{F}$かつ$\bigcup_nA_n\in\mathcal{G}$
すなわち$\bigcup_nA_n\in\mathcal{F}\cap\mathcal{G}$,
\end{itemize}
となる.従って,$\mathcal{F}\cap\mathcal{G}$は$X$上の$\sigma$-加法族である.
\item[(2)]
$\mathcal{F}\cup\mathcal{G}$が$X$上の$\sigma$-加法族とは限らない.
実際,
\[
X=\{1,2,3\},
\mathcal{F}=\{\varnothing,\{1\},\{2,3\},X\},
\mathcal{G}=\{\varnothing,\{2\},\{1,3\},X\}
\]
とすると,$\mathcal{F},\mathcal{G}$は集合$X$上の$\sigma$-加法族であるが,
\[
\mathcal{F}\cup\mathcal{G}=\{\varnothing,\{1\},\{2\},\{2,3\},\{1,3\},X\}
\]
は$\{1\}\cup\{2\}=\{1,2\}\notin\mathcal{F}\cup\mathcal{G}$より$\sigma$-加法族ではない.
\end{itemize}
\end{proof}

\begin{qst}
空でない集合$X$に対し,離散位相空間$(X,2^X)$を考える.
このとき,$X$の部分集合$A$が有限集合であることは,
$A$がコンパクトであるための必要十分条件であることを示せ.
\end{qst}
\begin{proof}
以下,$\{O_\lambda\}_{\lambda\in\Lambda}$を$A$の任意の開被覆とする.
$A$が$n$個の元$a_1,\cdots,a_n$からなる有限集合であると仮定すると,
各$i=1,\cdots,n$に対して$a_i\in O_{\lambda_i}$を満たす$\lambda_i\in\Lambda$を取れば
$\{O_{\lambda_i}\}_{i=1}^n$は$\{O_\lambda\}_{\lambda\in\Lambda}$の有限部分被覆となる.
よって$A$はコンパクトである.
逆に$A$がコンパクトであると仮定すると,
$A$の開被覆$\{\{a\}:a\in A\}$は有限部分被覆
$\{\{a_1\},\cdots,\{a_n\}:a_1,\cdots,a_n\in A\}$を持つ.
よって$A\subset\{a_1,\cdots,a_n\}$より$A$は有限集合である.
以上より,$X$の部分集合$A$が有限集合であることは,
$A$がコンパクトであるための必要十分条件である.
\end{proof}

\begin{qst}
$\{\lambda_n\}_{n=1}^\infty$を正の実数列で$\lambda>0$に収束するものとする.
各$i=1,2,\cdots$に対して,パラメータ$\lambda_i$を持つ指数分布を$\mu_i$とおく.
すなわち,$B\in\mathcal{B}(\mathbb{R})$に対して
\[ \mu_i(B)=\int_B\lambda_i\exp(-\lambda_ix)1_{[0,\infty)}(x)\,dx \]
であるとする.同様に,パラメータ$\lambda$を持つ指数分布を$\mu$とおく.
このとき,確率分布$\mu_n$は$\mu$に弱収束することを示せ.
\end{qst}
\begin{proof}
$f$を$\mathbb{R}$上の任意の有界連続関数とする.このとき
\[
\int_\mathbb{R}f(x)\,d\mu_n(x)
=\int_0^\infty f(x)\lambda_n\exp(-\lambda_nx)\,dx
=\int_0^\infty f\left(\frac{y}{\lambda_n}\right)\exp(-y)\,dy
\]
である.ここで,各$y\in[0,\infty)$および$n=1,2,\cdots$に対して
\[
\left|f\left(\frac{y}{\lambda_n}\right)\exp(-y)\right|\leq\exp(-y)\sup_{y\in[0,\infty)}|f(y)|,
\lim_{n\to\infty}f\left(\frac{y}{\lambda_n}\right)\exp(-y)
=f\left(\frac{y}{\lambda}\right)\exp(-y)
\]
であり,優収束定理の仮定を満たすから
\[
\lim_{n\to\infty}\int_\mathbb{R}f(x)\,d\mu_n(x)
=\lim_{n\to\infty}\int_0^\infty f\left(\frac{y}{\lambda_n}\right)\exp(-y)\,dy
=\int_0^\infty f\left(\frac{y}{\lambda}\right)\exp(-y)\,dy
=\int_\mathbb{R}f(x)\,d\mu(x)
\]
となる.よって,$\mu_n$は$\mu$に弱収束する.
\end{proof}

\end{document}
