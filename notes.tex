\documentclass{jsarticle}
\usepackage{amsmath, amssymb} % 数学記号のパッケージ

% 定理などの番号付けに関するパッケージとその設定
\usepackage{amsthm}
\theoremstyle{definition}
\newtheorem{qst}{問題}
\newtheorem{rem}[qst]{注意}
\renewcommand\proofname{\bf 解答} % proof環境の設定変更

\begin{document}

\begin{qst}
実数列$\{a_n\}_{n=1}^\infty$に対して
\[ \lim_{n\to\infty}\frac{a_n}{n}=0\Rightarrow\lim_{n\to\infty}a_n=0 \]
は成り立つか.成り立つ場合は証明し,成り立たない場合は反例を挙げよ.
\end{qst}
\begin{proof}
成り立たない.
実際,数列$\{1\}_{n=1}^\infty$が反例である.
\end{proof}

\begin{qst}
$f\colon\mathbb{R}\to\mathbb{R}$に対して
\[ f は有界かつ連続 \Rightarrow f は一様連続 \]
は成り立つか.成り立つ場合は証明し,成り立たない場合は反例を挙げよ.
\end{qst}
\begin{proof}
成り立たない.
実際,$f(x)=\sin{x^2}$とすると,$f$は有界かつ連続であるが,
任意の$\delta>0$に対して$n>\delta^{-2}$を満たす$n\geq1$をとれば,
$x=\sqrt{n\pi}, y=\sqrt{n\pi+\pi/2}$のとき
\[ |x-y|=\sqrt{n\pi+\frac{\pi}{2}}-\sqrt{n\pi}
=\frac{\pi/2}{\sqrt{n\pi+\pi/2}+\sqrt{n\pi}}
<\frac{\pi/2}{2\sqrt{n\pi}}<\frac{2}{2\sqrt{n\pi}}
=\frac{1}{\sqrt{n\pi}}<\frac{1}{\sqrt{n}}
<\delta \]
かつ
\[ |\sin{x^2}-\sin{y^2}|=\left|\sin{n\pi}-\sin\left(n\pi+\frac{\pi}{2}\right)\right|=1 \]
となるから,$f$は一様連続でない.
\end{proof}

\end{document}
