\documentclass{jsarticle}
\usepackage{amsmath, amssymb} % 数学記号のパッケージ

% 定理などの番号付けに関するパッケージとその設定
\usepackage{amsthm}
\theoremstyle{definition}
\newtheorem{qst}{問題}
\newtheorem{rem}[qst]{注意}
\renewcommand\proofname{\bf 解答} % proof環境の設定変更

\begin{document}

\begin{qst}
実数列$\{a_n\}_{n=1}^\infty$に対して
\[ \lim_{n\to\infty}\frac{a_n}{n}=0\Rightarrow\lim_{n\to\infty}a_n=0 \]
は成り立つか.成り立つ場合は証明し,成り立たない場合は反例を挙げよ.
\end{qst}
\begin{proof}
成り立たない.
実際,数列$\{1\}_{n=1}^\infty$が反例である.
\end{proof}

\begin{qst}
$f\colon\mathbb{R}\to\mathbb{R}$に対して
\[ f は有界かつ連続 \Rightarrow f は一様連続 \]
は成り立つか.成り立つ場合は証明し,成り立たない場合は反例を挙げよ.
\end{qst}
\begin{proof}
成り立たない.
実際,$f(x)=\sin{x^2}$とすると,$f$は有界かつ連続であるが,
任意の$\delta>0$に対して$n>\delta^{-2}$を満たす$n\geq1$をとれば,
$x=\sqrt{n\pi}, y=\sqrt{n\pi+\pi/2}$のとき
\[ |x-y|=\sqrt{n\pi+\frac{\pi}{2}}-\sqrt{n\pi}
=\frac{\pi/2}{\sqrt{n\pi+\pi/2}+\sqrt{n\pi}}
<\frac{\pi/2}{2\sqrt{n\pi}}<\frac{2}{2\sqrt{n\pi}}
=\frac{1}{\sqrt{n\pi}}<\frac{1}{\sqrt{n}}
<\delta \]
かつ
\[ |\sin{x^2}-\sin{y^2}|=\left|\sin{n\pi}-\sin\left(n\pi+\frac{\pi}{2}\right)\right|=1 \]
となるから,$f$は一様連続でない.
\end{proof}

\begin{qst}
実数列$\{a_n\}_{n=1}^\infty$および$a\in\mathbb{R}$に対して
\[ \lim_{n\to\infty}a_n=a\Leftrightarrow\{a_n\}_{n=1}^\infty の任意の部分列が a に収束する部分列を持つ \]
が成り立つことを証明せよ.
\end{qst}
\begin{proof}
($\Rightarrow$)
$\{a_n\}_{n=1}^\infty$の任意の部分列が$a$に収束することから従う.

($\Leftarrow$)
対偶を証明する.
すなわち,ある$\varepsilon>0$が存在して,任意の$N\geq1$に対して
$n\geq N$かつ$|a_n-a|\geq\varepsilon$を満たす$n\geq1$が存在すると仮定する.
各$N\geq1$に対して,$n\geq N$かつ$|a_n-a|\geq\varepsilon$を満たす$n\geq1$を
ひとつ選び$n(N)$とおく.さらに,$A:=\{n(N):N\geq1\}$とし,
数列$\{k(m)\}_{m=1}^\infty$を次のように定める:
$k(1):=\min A$,$m\geq1$に対し$k(m+1):=\min(A\setminus\{k(1),\cdots,k(m)\})$.
このとき,部分列$\{a_{k(m)}\}_{m=1}^\infty$はすべての$m\geq1$に対して
$|a_{k(m)}-a|\geq\varepsilon$を満たすから,$a$に収束する部分列を持たない.
\end{proof}

\begin{qst}
$\mathbb{N}$の空でない任意の部分集合$A$は最小値を持つことを証明せよ.
\end{qst}
\begin{proof}
(1)
$A$が有限集合であるとき,$n:=\sharp{A}$に関する帰納法で証明する.
$n=1$のときは$a\in A$が最小値となり,
$n=2$のときは,相異なる$a,b\in A$に対して$\min\{a,b\}$が最小値となる.
$n=k\geq2$のとき,$A$の最小値が存在すると仮定する.
$\sharp{A}=k+1$とすると,$a\in A$をひとつ選んで
$A=\{a\}\cup B$,$\sharp{B}=k$と表せる.
このとき$a\leq\min B$ならば$\min A=a$,そうでなければ$\min A=\min B$となる.
よって,$n=k+1$のときも$A$の最小値が存在する.

(2)
$A$が無限集合であるとき,
$n\in A$をひとつ取り$A_n:=\{a\in A:a<n\}$とおく.
$A_n=\varnothing$ならば$n=\min A$である.
$A_n\neq\varnothing$のとき,$A_n$は有限集合であるから
(1)より$\min A_n$が存在し,$\min A=\min A_n$である.
\end{proof}

\end{document}
