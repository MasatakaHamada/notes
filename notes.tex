\documentclass{jsarticle}
\usepackage{amsmath, amssymb} % 数学記号のパッケージ

% 定理などの番号付けに関するパッケージとその設定
\usepackage{amsthm}
\theoremstyle{definition}
\newtheorem{qst}{問題}
\newtheorem{rem}[qst]{注意}
\renewcommand\proofname{\bf 解答} % proof環境の設定変更

\begin{document}

\begin{qst}
実数列$\{a_n\}_{n=1}^\infty$に対して
\[ \lim_{n\to\infty}\frac{a_n}{n}=0\Rightarrow\lim_{n\to\infty}a_n=0 \]
は成り立つか.成り立つ場合は証明し,成り立たない場合は反例を挙げよ.
\end{qst}
\begin{proof}
成り立たない.
実際,数列$\{1\}_{n=1}^\infty$が反例である.
\end{proof}

\begin{qst}
$f\colon\mathbb{R}\to\mathbb{R}$に対して
\[ f は有界かつ連続 \Rightarrow f は一様連続 \]
は成り立つか.成り立つ場合は証明し,成り立たない場合は反例を挙げよ.
\end{qst}
\begin{proof}
成り立たない.
実際,$f(x)=\sin{x^2}$とすると,$f$は有界かつ連続であるが,
任意の$\delta>0$に対して$n>\delta^{-2}$を満たす$n\geq1$をとれば,
$x=\sqrt{n\pi}, y=\sqrt{n\pi+\pi/2}$のとき
\[ |x-y|=\sqrt{n\pi+\frac{\pi}{2}}-\sqrt{n\pi}
=\frac{\pi/2}{\sqrt{n\pi+\pi/2}+\sqrt{n\pi}}
<\frac{\pi/2}{2\sqrt{n\pi}}<\frac{2}{2\sqrt{n\pi}}
=\frac{1}{\sqrt{n\pi}}<\frac{1}{\sqrt{n}}
<\delta \]
かつ
\[ |\sin{x^2}-\sin{y^2}|=\left|\sin{n\pi}-\sin\left(n\pi+\frac{\pi}{2}\right)\right|=1 \]
となるから,$f$は一様連続でない.
\end{proof}

\begin{qst}
実数列$\{a_n\}_{n=1}^\infty$および$a\in\mathbb{R}$に対して
\[ \lim_{n\to\infty}a_n=a\Leftrightarrow\{a_n\}_{n=1}^\infty の任意の部分列が a に収束する部分列を持つ \]
が成り立つことを証明せよ.
\end{qst}
\begin{proof}
($\Rightarrow$)
$\{a_n\}_{n=1}^\infty$の任意の部分列が$a$に収束することから従う.

($\Leftarrow$)
対偶を証明する.
すなわち,ある$\varepsilon>0$が存在して,任意の$N\geq1$に対して
$n\geq N$かつ$|a_n-a|\geq\varepsilon$を満たす$n\geq1$が存在すると仮定する.
各$N\geq1$に対して,$n\geq N$かつ$|a_n-a|\geq\varepsilon$を満たす$n\geq1$を
ひとつ選び$n(N)$とおく.さらに,$A:=\{n(N):N\geq1\}$とし,
数列$\{k(m)\}_{m=1}^\infty$を次のように定める:
$k(1):=\min A$,$m\geq1$に対し$k(m+1):=\min(A\setminus\{k(1),\cdots,k(m)\})$.
このとき,部分列$\{a_{k(m)}\}_{m=1}^\infty$はすべての$m\geq1$に対して
$|a_{k(m)}-a|\geq\varepsilon$を満たすから,$a$に収束する部分列を持たない.
\end{proof}

\begin{qst}
$\mathbb{N}$の空でない任意の部分集合$A$は最小値を持つことを証明せよ.
\end{qst}
\begin{proof}
(1)
$A$が有限集合であるとき,$n:=\sharp{A}$に関する帰納法で証明する.
$n=1$のときは$a\in A$が最小値となり,
$n=2$のときは,相異なる$a,b\in A$に対して$\min\{a,b\}$が最小値となる.
$n=k\geq2$のとき,$A$の最小値が存在すると仮定する.
$\sharp{A}=k+1$とすると,$a\in A$をひとつ選んで
$A=\{a\}\cup B$,$\sharp{B}=k$と表せる.
このとき$a\leq\min B$ならば$\min A=a$,そうでなければ$\min A=\min B$となる.
よって,$n=k+1$のときも$A$の最小値が存在する.

(2)
$A$が無限集合であるとき,
$n\in A$をひとつ取り$A_n:=\{a\in A:a<n\}$とおく.
$A_n=\varnothing$ならば$n=\min A$である.
$A_n\neq\varnothing$のとき,$A_n$は有限集合であるから
(1)より$\min A_n$が存在し,$\min A=\min A_n$である.
\end{proof}

\begin{qst}
$x\in\mathbb{R}$に対して
\[ \delta_x(B):=1_B(x)=\begin{cases}1 & (x\in B) \\ 0 & (x\notin B)\end{cases},B\in\mathcal{B}(\mathbb{R}) \]
と定める.このとき,$\delta_x$は$(\mathbb{R},\mathcal{B}(\mathbb{R}))$上の確率測度であることを示せ.
なお,この$\delta_x$は$x$を台とするディラック測度と呼ばれる.
\end{qst}
\begin{proof}
定義より$\delta_x(\mathbb{R})=1$かつ$B\in\mathcal{B}(\mathbb{R})$に対して$\delta_x(B)\geq0$である.
$B_1,B_2,\cdots\in\mathcal{B}(\mathbb{R})$を$i\neq j$に対して$B_i\cap B_j=\varnothing$を満たすものとする.
すべての$i$に対して$x\notin B_i$であるとき,
\[ \delta_x\left(\bigcup_{i=1}^\infty B_i\right)=0=\sum_{i=1}^\infty\delta_x(B_i) \]
である.また,ある$N$が存在して$x\in B_N$であるとき,$B_1,B_2,\cdots$の定義よりそのような$N$はただひとつであるから
\[ \delta_x\left(\bigcup_{i=1}^\infty B_i\right)=1=\delta_x(B_N)=\sum_{i=1}^\infty\delta_x(B_i) \]
である.
\end{proof}

\begin{qst}[未解決]
$f\colon[0,\infty)\to\mathbb{R}$を$(0,\infty)$上連続な関数とする.このとき
\[ \lim_{r\to+0, r\in\mathbb{Q}}f(r)=f(0) \Rightarrow \lim_{x\to+0}f(x)=f(0) \]
が成り立つことを証明せよ(従って,$f$は$[0,\infty)$上連続となる).
\end{qst}
\begin{proof}[\textbf{誤答例}]
0に収束する任意の非負実数列$\{x_n\}_{n=1}^\infty$に対して,
$\{f(x_n)\}_{n=1}^\infty$が$f(0)$に収束することを示す.
そのためには,$\{f(x_n)\}_{n=1}^\infty$の任意の部分列が
$f(0)$に収束する部分列を有することを示せばよい.
そこで,$\{f(x_k)\}_{k=1}^\infty$を$\{f(x_n)\}_{n=1}^\infty$の任意の部分列とする.
このとき,
\textbf{$\{x_k\}_{k=1}^\infty$から有理数である項を取り出して得られる部分列を$\{x_{k_l}\}_{l=1}^\infty$とする}と,
仮定より$\{f(x_{k_l})\}_{l=1}^\infty$は$f(0)$に収束する.
よって,$\{f(x_k)\}_{k=1}^\infty$に対して$f(0)$に収束する部分列が得られたので,結論が従う.
\end{proof}
\begin{proof}[\textbf{誤答例}]
$\varepsilon>0$とする.
仮定より,ある$\delta_1>0$が存在して,$0<r<\delta_1$を満たす任意の$r\in\mathbb{Q}$に対して
$|f(r)-f(0)|<\varepsilon/2$を満たす.
以下,このような$\delta_1$を固定し,さらに$0<r<\delta_1$を満たす$r\in\mathbb{Q}$を固定する.
$f$は$(0,\infty)$上連続であるから,ある$\delta_2>0$が存在して,
$|x-r|<\delta_2$を満たす任意の$x\in(0,\infty)$に対して$|f(x)-f(r)|<\varepsilon/2$を満たす.
このとき,\textbf{$\delta:=\min\{\delta_2-\delta_1,0\}$とおく}と,
$0<x<\delta$を満たす任意の$x\in(0,\infty)$に対して
$|x-r|\leq x+r<\delta+\delta_1\leq\delta_2$となる.よって
$|f(x)-f(0)|\leq|f(x)-f(r)|+|f(r)-f(0)|<\varepsilon/2+\varepsilon/2=\varepsilon$である.
従って,結論が成り立つ.
\end{proof}

\begin{qst}
区間$I$上の連続関数列$\{f_n\}_{n=1}^\infty$と
$I$上の連続関数$f$に対して,次の2つの条件は同値であることを示せ.
\begin{enumerate}
\item$\{f_n\}_{n=1}^\infty$は$n\to\infty$のとき$I$上$f$に一様収束する.
\item$\displaystyle\sup_{x\in I}|f_n(x)-f(x)|$は$n\to\infty$のとき0に収束する.
\end{enumerate}
\end{qst}
\begin{proof}
(1$\Rightarrow$2)
仮定より,任意の$\varepsilon>0$に対してある$N$が存在して,
$n\geq N$ならば任意の$x\in I$に対して$|f_n(x)-f(x)|<\varepsilon$が成り立つ.
よって,$n\geq N$ならば$\displaystyle\sup_{x\in I}|f_n(x)-f(x)|<\varepsilon$となる.
従って,$\displaystyle\sup_{x\in I}|f_n(x)-f(x)|$は$n\to\infty$のとき0に収束する.

(2$\Rightarrow$1)
仮定より,任意の$\varepsilon>0$に対してある$N$が存在して,
$n\geq N$ならば$\displaystyle\sup_{x\in I}|f_n(x)-f(x)|<\varepsilon$が成り立つ.
よって,$n\geq N$ならば任意の$x\in I$に対して
$|f_n(x)-f(x)|\leq\displaystyle\sup_{x\in I}|f_n(x)-f(x)|<\varepsilon$が成り立つ.
従って,$\{f_n\}_{n=1}^\infty$は$n\to\infty$のとき$I$上$f$に一様収束する.
\end{proof}

\begin{qst}
$\|\cdot\|$をノルムに持つバナッハ空間$V$上のコーシー列$\{v_n\}$と
任意の$\varepsilon>0$に対して,
\[ m,n\geq N\Rightarrow\|v_m-v_n\|<\varepsilon \]
を満たす自然数$N$を選ぶ.$\{v_n\}$の極限を$v$とするとき,
\[ n\geq N\Rightarrow\|v-v_n\|\leq\varepsilon \]
が成り立つことを証明せよ.
\end{qst}
\begin{proof}
任意の$\varepsilon'>0$に対して,
$m\geq M$ならば$\|v_m-v\|<\varepsilon'$となるような自然数$M$が存在する.
よって,$m,n\geq\max\{M,N\}$ならば
$\|v-v_n\|=\|(v_m-v_n)+(v-v_m)\|\leq\|v_m-v_n\|+\|v_m-v\|<\varepsilon+\varepsilon'$
が成り立つ.
ゆえに,$n\geq N$ならば$\|v-v_n\|<\varepsilon+\varepsilon'$が
任意の$\varepsilon'>0$に対して成り立つ.
このとき,もし$\|v-v_n\|>\varepsilon$であるとすると,
$\varepsilon<\delta<\|v-v_n\|$なる実数$\delta$が存在する.
この$\delta$に対して$\|v-v_n\|>\delta=\varepsilon+(\delta-\varepsilon)$となり,
$\|v-v_n\|<\varepsilon+\varepsilon'$が
任意の$\varepsilon'>0$に対して成り立つことに矛盾する.
従って,$n\geq N\Rightarrow\|v-v_n\|\leq\varepsilon$が成り立つ.
\end{proof}

\begin{qst}
実数列$\{a_n\}$と$a\in\mathbb{R}$に対して,次の2つの条件は同値であることを示せ.
\begin{enumerate}
\item$\displaystyle\lim_{n\to\infty}a_n=a$である.
すなわち,任意の$\varepsilon>0$に対してある$N\in\mathbb{N}$が存在して,
$n\geq N$ならば$|a_n-a|<\varepsilon$が成り立つ.
\item 任意の$\varepsilon>0$に対してある$N\in\mathbb{N}$が存在して,
$n\geq N$ならば$|a_n-a|\leq\varepsilon$が成り立つ.
\end{enumerate}
\end{qst}
\begin{proof}
(1$\Rightarrow$2)
$|a_n-a|<\varepsilon$ならば$|a_n-a|\leq\varepsilon$であることから従う.

(2$\Rightarrow$1)
$\varepsilon>0$とする.
2より,ある$N\in\mathbb{N}$が存在して,
$n\geq N$ならば$|a_n-a|\leq\varepsilon/2$が成り立つ.
このとき,$\varepsilon/2<\varepsilon$より$|a_n-a|<\varepsilon$である.
\end{proof}

\begin{qst}
集合$X$および$X$の部分集合列$A_1,A_2,\cdots$に対して次が成り立つことを示せ.
\[ 1_{\limsup_{n\to\infty}A_n}(x)=\limsup_{n\to\infty}1_{A_n}(x),
1_{\liminf_{n\to\infty}A_n}(x)=\liminf_{n\to\infty}1_{A_n}(x) \]
\end{qst}
\begin{proof}
以下
$F=\displaystyle\limsup_{n\to\infty}A_n$,
$G=\displaystyle\liminf_{n\to\infty}A_n$
とおく.
$x\in F$のとき,任意の$m\geq1$に対してある$n\geq m$が存在して$x\in A_n$を満たすから,
\[ \limsup_{n\to\infty}1_{A_n}(x)=\inf_{m\geq1}\sup_{n\geq m}1_{A_n}(x)=1 \]
である.
$x\notin F$のとき,ある$m\geq1$が存在して$n\geq m$ならば$x\notin A_n$となるから,
\[ \limsup_{n\to\infty}1_{A_n}(x)=\inf_{m\geq1}\sup_{n\geq m}1_{A_n}(x)=0 \]
である.
$x\in G$のとき,ある$m\geq1$が存在して$n\geq m$ならば$x\in A_n$となるから,
\[ \liminf_{n\to\infty}1_{A_n}(x)=\sup_{m\geq1}\inf_{n\geq m}1_{A_n}(x)=1 \]
である.
$x\notin G$のとき,任意の$m\geq1$に対してある$n\geq m$が存在して$x\notin A_n$を満たすから,
\[ \liminf_{n\to\infty}1_{A_n}(x)=\sup_{m\geq1}\inf_{n\geq m}1_{A_n}(x)=0 \]
である.以上より
\[ 1_{\limsup_{n\to\infty}A_n}(x)=\limsup_{n\to\infty}1_{A_n}(x),
1_{\liminf_{n\to\infty}A_n}(x)=\liminf_{n\to\infty}1_{A_n}(x) \]
が成り立つ.
\end{proof}

\begin{qst}
$X,X_1,X_2,\cdots$を2乗可積分な確率変数とする.
$X_n$が$X$に2次平均収束するとき,$X_n^2$の期待値は$X^2$の期待値に収束することを示せ.
すなわち
\[ \lim_{n\to\infty}E[(X_n-X)^2]=0\Rightarrow\lim_{n\to\infty}E[X_n^2]=E[X^2] \]
であることを示せ.
\end{qst}
\begin{proof}
$E[(X_n-X)^2]\to0$より$\|X_n-X\|_2\to0$であるから
$|\|X_n\|_2-\|X\|_2|\leq\|X_n-X\|_2\to0$
ゆえ$\|X_n\|_2\to\|X\|_2$すなわち$E[X_n^2]\to E[X^2]$が従う.
\end{proof}

\begin{qst}
$(\Omega,\mathcal{F},P)$を確率空間とする.
$A,B\in\mathcal{F}$に対して,$P(A)=1$ならば$P(A\cap B)=P(B)$であることを示せ.
\end{qst}
\begin{proof}
$P(A)=1$より$P(A^c)=0$であるから,
特に$P(A^c\cap B)\leq P(A^c)=0$より$P(A^c\cap B)=0$である.
よって$P(B)=P(A\cap B)+P(A^c\cap B)=P(A\cap B)$となる.
\end{proof}

\begin{qst}
$(\Omega,\mathcal{F},P)$を確率空間,$X$を可積分確率変数とし,
$A\in\mathcal{F}$は$P(A)=1$を満たすとする.
このとき$E(X)=E(X;A)$が成り立つことを示せ.
\end{qst}
\begin{proof}
$P(A)=1$より$P(A^c)=0$であり,零集合上の積分は0であるから$E(X;A^c)=0$である.
よって$E(X)=E(X;A)+E(X;A^c)=E(X;A)$となる.
\end{proof}

\begin{qst}
$(X,\mathcal{M})$を可測空間とする.
$X$の空でない部分集合$A$に対して
\[ \mathcal{M}_A=\{A\cap M:M\in\mathcal{M}\} \]
と定めるとき,$(A,\mathcal{M}_A)$は可測空間であることを示せ.
\end{qst}
\begin{proof}
$A\cap\varnothing=\varnothing$,
$A\cap X=A$ゆえ$\varnothing,A\in\mathcal{M}_A$である.
また,$F=A\cap M\in\mathcal{M}_A$のとき,
$A=(A\cap M)\cup(A\cap M^c)$であることに注意すると
$A\setminus F=A\setminus(A\cap M)=A\cap M^c$ゆえ
$A\setminus F\in\mathcal{M}_A$である.
さらに,$n=1,2,\cdots$に対して$F_n=A\cap M_n\in\mathcal{M}_A$とするとき,
$\bigcup_nF_n=\bigcup_n(A\cap M_n)=A\cap\bigcup_nM_n$ゆえ
$\bigcup_nF_n\in\mathcal{M}_A$である.
以上より,$(A,\mathcal{M}_A)$は可測空間である.
\end{proof}

\end{document}
